%\setlength{\belowdisplayskip}{0pt} \setlength{\belowdisplayshortskip}{0pt}
%\setlength{\abovedisplayskip}{0pt} \setlength{\abovedisplayshortskip}{0pt}

\subsection{Frames of Reference}
%
In fluid dynamics, there are two frames of reference to observe fluid motion: Eulerian and Lagrangian.
%
With the Eulerian frame of reference, the observer is in a fixed position.
%
With the Lagrangian frame of reference, the observer is attached to a fluid parcel and is moving through space and time.

%
When a flow field is stored in an Eulerian representation, it is typically done by means of its velocity field.
%
A velocity field $v$ is a time-dependent vector field that maps each point $x\in \mathbb R^d$ in space to the velocity of the flow field for a given time $t\in \mathbb R$
%
\begin{eqnarray}
{v} : \mathbb R^d \times \mathbb R \to \mathbb R^d,\; x,t \mapsto v(x,t)
\end{eqnarray}

%In a practical setting, the vector field is defined over a fixed, discrete mesh and represents the state of the flow field at a specific instant of time or time slice, i.e., at a specific simulation time and cycle.
%
In a practical setting, a flow field at a specific time/cycle is defined as a vector data on a fixed, discrete mesh.
%
Time-varying flow is represented as a collection of such data over a variety times/cycles.


When a flow field is stored in a Lagrangian representation, it is done by means of its flow map $F_{t_0}^{t}$.
%
The flow map is comprised of the starting positions of massless particles $x_0$ at time $t_0$ and their respective trajectories that are interpolated using the time-dependent vector field.
%
Mathematically, a flow map is defined as the mapping
\begin{eqnarray}
F_{t_0}^{t}(x_0):\mathbb R \times \mathbb R \times \mathbb R^d \to \mathbb R^d,\; t \times t_0 \times x_0 \mapsto F_{t_0}^{t}(x_0) = x(t)
\end{eqnarray}
%
of initial values $x_0$ to the solutions of the ordinary differential equation
%
\begin{eqnarray}
\frac{d}{dt}x(t) = v(x(t),t)
\end{eqnarray}

In a practical setting, the flow map is represented as sets of particle trajectories calculated in the time interval $[t_0,t]\subset \mathbb R$.
%
The stored information, encoded in the form of known particle trajectories (i.e., a Lagrangian representation), encodes the behavior of the time-dependent vector field over an interval of time.
%
%
%Although the frames of reference are theoretically equivalent~\cite{bujack2015lagrangian}, their application in practical settings varies. 
%
%Our interest with this study is the practical setting, specifically for the \textbf{EUS} problem.
%

\subsection{Lagrangian Analysis}
Within the vector field analysis and visualization community, Lagrangian methods have been increasingly researched in the past decade. 
%
The interest in Lagrangian representations is motivated by temporal sparsity of data, i.e., simulations store data less frequently to avoid high storage costs.
%
Lagrangian representations are expected to perform better than the Eulerian counterparts in sparse temporal settings, and the key intuition behind this is that the Lagrangian representations captures the behavior of the flow field over an interval of time, as opposed to the state at a single time slice.
%
However, in addition to the temporal frequency of the data available, the nature of the vector field can also impact the quality of reconstruction.
%
Our study explores this aspect by investigating the benefits and limitations of using Lagrangian representations for time-varying vector fields of unexplored applications.

Prior works have advanced research along various axes of Lagrangian analysis.
%
Agranovsky et al.~\cite{agranovsky2014improved} and Sane et al.~\cite{sane2018revisiting} evaluated reduced Lagrangian representations. 
%
A shortcoming of these studies, is the use of a single average error for the reconstruction, thus only providing a preliminary evaluation of efficacy. 
%
Further, experiments were conducted in a theoretical in situ setting, i.e., files were loaded from disk rather than integration with a simulation. 
%
Such a theoretical set up fails to capture the cost of invoking in situ processing every cycle in a tightly coupled system.
%
%Recently, Pascal et al.~\cite{envirvis.20171099,siegfried2019tropical} use embedded routines to compute reduced Lagrangian data in order to explore coastal upwelling activity and visualize a derived scalar field representing trajectory density.
%
%As such, prior works considering reduced Lagrangian representations have been restricted to analytical, ocean or climate data.
%

With respect to extracting trajectories in situ, Sane et al.~\cite{sane2019interpolation} explored the extraction of variable duration trajectories and consequently proposed a post hoc interpolation scheme to reduce reconstruction error by evaluating neighborhoods across interpolations.
%
Rapp et al.~\cite{rapp2019void} applied their void-and-cluster sampling technique to identify a representative set of scattered samples and found that it performs better than random sampling.
%
To address scalability challenges, Sane et al.~\cite{sane2020scalable} explored an accuracy-performance tradeoff and demonstrated the use of a communication-free model that only stored trajectories that remain within the rank domain during the interval of computation.
%

Several prior works have presented strategies for post hoc reconstruction.
%
Hlawatsch et al.~\cite{hlawatsch2011hierarchical} proposed a hierarchical reconstruction scheme that can improve accuracy, but relies on access to data across multiple time intervals.
%
Chandler et al.~\cite{chandler2015interpolation} proposed a modified k-d tree as a search structure for a Lagrangian data extracted from an SPH simulation.
%
Further, Chandler et al.~\cite{chandler2016analysis} identified correlations between Lagrangian-based interpolation error and divergence in the flow field.
%
Bujack et al.~\cite{bujack2015lagrangian} evaluated the use of parameter curves to fit interpolated pathline points to improve the aesthetic of trajectories calculated using Lagrangian data.
%
Hummel et al.~\cite{hummel2016error} provided theoretical error bounds for error propagation and accumulation that can occur when calculating trajectories using Lagrangian representations. 
%
Most recently, Jakob et al.~\cite{Jakob20} demonstrated the use of DNNs to up-sample reduced Lagrangian representations of 2D flow fields. 


\subsection{Time-Varying Vector Field Reduction}
%Within the vector field analysis and visualization community, Lagrangian methods have been increasingly used in the past decade.
%
%Lagrangian coherent structures (LCS) are a popular technique to visualize attracting and repelling surfaces and were introduced by Haller et al~\cite{haller2001distinguished, haller2000lagrangian, haller2000finding}.
%
%The interest in the technique led to multiple efforts that were aimed at accelerating the computation and visualization of LCS~\cite{garth2007efficient,garth2009visualization,sadlo2007efficient,sadlo2011time}.
%
%LCS have also been used for uncertain transient vector field visualization by Guo et al.~\cite{guo2016finite}.
%

Although a few studies have shown that Eulerian representations can be susceptible to temporal sparsity~\cite{costa2004lagrangian}\cite{Qin2014}\cite{agranovsky2014improved}\cite{sane2018revisiting}, temporal subsampling remains the widely used solution to limit data storage.
%
Our study contributes to this body of work by using temporal subsampling as a reference point in our investigation.
%
Multiple works have proposed single time step vector field reduction strategies while maintaining an Eulerian representation.
%
%Lodha et al.~\cite{lodha2000topology} controlled the compression of similar vectors into single vectors representing larger area.
%
%Further, Lodha et al.~\cite{lodha2003topology} proposed a top-down topology preserving compression technique.
%
%Theisel et al.~\cite{theisel2003combining} computed critical points and viewed the task as a mesh reduction, and later provided a threshold to filter important features~\cite{theisel2003compression}.
%
%With the objective of highlighting temporal features of the vector field, Tong et al.~\cite{tong2012salient} compressed the total amount of data steps stored by identifying key time steps.
%
Although these techniques could be used to reduce data and store more frequently, these approaches don't inherently address the challenge of increasing temporal sparsity.
%

In a recent large-scale tornadic supercell thunderstorm study~\cite{atmos10100578}, Leigh Orf modified the I/O code to use a hierarchical data format and lossy floating-point compression via ZFP~\cite{lindstrom2006fast}.
%
ZFP provides dynamic accuracy control by allowing the user to specify a maximum amount of deviation.
%
The author notes, although ZFP is effective for scalar fields that do not require differentiation during post hoc analysis, to maintain accurate vector field reconstruction only a very small value of deviation can be chosen for each component of velocity.
%
Thus, ZFP allows a limited amount of compression to time-varying vector field data without introducing significant uncertainty to post hoc analysis. 
%
The technique provided an average reduction of 30\% of total uncompressed vector field data, with regions of high gradient resulting in less compression. 
%
%Overall, the scientist referred to the use of lossy compression as unfortunate but necessary.

%In this paper, using the same temporal sampling frequency as the Lagrangian representations, we measure the reconstruction accuracy achieved by temporal subsampling for reference.
%Although these techniques are valuable, they do not sufficiently address the challenge of increasing temporal sparsity.
%
%\fix{I think we need a better statement here --- could reduce to get higher temporal sparsity.  And yet don't want to compare with that.}
%For a baseline comparison in our empirical study, we store data in an Eulerian representation at full resolution and use temporal subsampling.
%
%This paper presents an empirical study of the in situ encumbrance and performance tradeoffs when executing \textit{in situ Lagrangian analysis} in a practical setting, i.e., in situ on a supercomputer.

