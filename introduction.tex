%Top}ics for introduction / related work:
%1. One of the challenges faced by scientists is the data analysis and visualization of the large amounts of data generated by simulation codes on HPC resources.
%2. In situ processing offers a potential solution by circumventing the need to store all the raw data. 
%3. Most in situ analysis tasks operate infrequently or are triggered. Computing Lagrangian flow maps is unlike other analysis tasks and requires computation every cycles of the simulation - this places an overhead on the simulation code and raises questions of viability. 
%4. Lagrangian analysis has been extensively studied for ocean modeling in an offline setting and now efforts to support online analysis are being made.
%5. Reduced Lagrangian flow maps have been shown to be superior to traditional Eulerian subsampling for primarily analytical data sets, in theoretical settings or 2D flows, and compared using a single average error across all post hoc interpolated particles. 
%6. We believe reduced Lagrangian flow maps need to be explored for more real world time-varying vector fields to better understand their applicability, and more closely evalauted - both quantitatively and qualitatively - to understand efficacy characteristics.
%7. We strongly believe Lagrangian flow maps need to be evaluated in representative settings - integrated with a simulation code and executing on a supercomputer - to provide insight into viability.
%8. Given reduced Lagrangian flow maps are a new paradigm - it opens research opportunities along multiple axes - namely, sampling strategy, post hoc interpolation, ease of use (integration), performance (viability) and applicability (configuration settings, vector field type, etc). 
%9. Several works have advanced research along these axes - but none have considered viability in relation to simulation execution times or application to seismology or cosmology vector fields. Demonstration in this form we believe encourages wider adoption of the paradigm beyond the theoretical level.
%
%
%
%
High-performance computing resources play a key role in advancing computational science by enabling modeling of scientific phenomena at high spatiotemporal resolutions.
%
%Although HPC enables modeling of scientific phenomena at high spatiotemporal resolutions, the total data generated is prohibitively large.
%
A challenge with regard to studying the output of a simulation is the prohibitively large size of the total data generated.
%
Compromise in the form of storing a subset of the data can impact the extent and accuracy of subsequent post hoc exploratory analysis and visualization.
%
In particular, for accurate time-varying vector field analysis and visualization, access to the full spatiotemporal resolution is required.
%
Since storing the entire simulation output is expensive, scientists resort to temporal subsampling or lossy compression, and often limit analysis to individual time slices.
%
An emerging paradigm to address large data challenges is the use of in situ processing to perform runtime analysis/visualization or data reduction to support exploratory post hoc analysis.
%
%

%In situ Lagrangian analysis is an emerging paradigm to enable post hoc exploration of time-varying vector fields.
% and has presented research opportunities along multiple orthogonal axes.
%
Lagrangian analysis is a powerful tool to study time-varying vector fields and is widely employed for ocean modeling applications~\cite{VANSEBILLE201849}.
%
The notion of calculating a Lagrangian representation or \textit{flow map}, i.e., sets of particle trajectories, ``online'' (in situ) for ``offline'' (post hoc) exploration was first proposed by Vries et al.~\cite{vries2001calculating} for an ocean modeling simulation.
%
Figure~\ref{fig:sample} illustrates the approach.
%
%The black trajectories are first extracted in situ and are referred to as \textit{basis} trajectories.
%
%They are used to reconstruct the red trajectory post hoc.
%
%The end location of the red trajectory deviates by a margin of error from the ground truth and is the result of using a Lagrangian-based advection scheme \textit{L}, i.e., a technique to interpolate flow maps.
%
%The quality of reconstruction depends on the vector field as well as configuration specifics such as sampling strategy and frequency of storage. 
%
More recently, multiple works have advanced Lagrangian research along axes such as strategies for in situ extraction of reduced Lagrangian representations~\cite{agranovsky2014improved}\cite{rapp2019void}\cite{sane2020scalable}, post hoc reconstruction~\cite{chandler2015interpolation}\cite{sane2019interpolation}\cite{Jakob20}, and theoretical error analysis~\cite{bujack2015lagrangian}\cite{chandler2016analysis}\cite{hummel2016error}.
%
%Although the effectiveness of Lagrangian representations for any possible time-varying vector fields that can be produced by a scientific simulation remains an open question, prior theoretical demonstration on analytical, SPH, climate and ocean modeling data, and practical application in ocean activity analysis~\cite{envirvis.20171099}\cite{siegfried2019tropical}, has provided encouraging results.
%
%the development of runtime in situ infrastructure for multiphysics simulations has enabled the straightforward extraction of accurate Lagrangian representations via APIs.
%

An open challenge for time-varying vector field exploration is predicting the uncertainty and variability in efficacy for different analysis techniques.
%
Although the effectiveness of Lagrangian representations for any possible time-varying vector fields that can be produced by a scientific simulation remains an open question, prior theoretical demonstration on vector fields including analytical, SPH, climate and ocean modeling data, and practical application in ocean activity analysis~\cite{envirvis.20171099}\cite{siegfried2019tropical}, has provided encouraging results.
%
%This is true for prior Lagrangian research.
%
Using Lagrangian representations, the quality of reconstruction of a time-varying vector field depends on the vector field itself, as well as configuration specifics such as sampling strategy and frequency of storage.
%
Thus, to investigate the potential benefits of Lagrangian representations for a broader range of applications and to gauge its viability in practice, we leverage the recent developments of runtime in situ infrastructure that enable the straightforward extraction via APIs to study Lagrangian representations for cosmology and seisomology applications.   
%Thus, to broade
%Thus, the application of Lagrangian representations to a broader range of real-world simulation vector fields, and in practice, is of keen interest to us.
%
%Considering reduced Lagrangian representations are currently in their early adoption phase, 
%Although prior theoretical evaluations have considered a limited set of applications including analytical, SPH, climate and ocean modeling data, and practical applications are limited to ocean modeling~\cite{envirvis.20171099}\cite{siegfried2019tropical}, .
%
%Further, prior works have also typically considered theoretical in situ environments, i.e., loaded data sets from disk. 
%
%Thus, an existing barrier for wider adoption is the lack of data points from a broader range of real-world simulation vector fields and the viability of the technique in practice.
%


%existing evaluations have considered a limited set of applications
%
%
%
%
%Existing research 
%%
%%With studies of reduced Lagrangian representations still in relatively early phases of adoption, its application has been demonstrated on a limited set of applications including analytical, SPH, climate and ocean modeling simulations.
%%
%%More recently, Pascal et al.~\cite{envirvis.20171099,siegfried2019tropical} used embedded routines to compute reduced Lagrangian data in order to explore coastal upwelling activity and visualize a derived scalar field representing trajectory density.
%
%
%In these works, since the quality of reconstruction has depended on the vector field as well as configuration specifics such as sampling strategy and frequency of storage
%
%Across this body of work, 
%
%A common thread in many of these studies has been efficacy characteris Lagrangian representations perform
%
% the quality of reconstruction depends on the vector field as well as configuration specifics such as sampling strategy and frequency of storage. 
%
%
%
%
%Evaluations of reduced Lagrangian representations, have shown that 
%
%the quality of reconstruction depends on the vector field as well as configuration specifics such as sampling strategy and frequency of storage. 
%
%they have a limited set of applications including analytical, SPH, and climate and ocean modeling simulations. 
%Although these works have produced increasingly advanced techniques for Lagrangian analysis, they have a limited set of applications including analytical, SPH, and climate and ocean modeling simulations. 
%
%
%an open question is whether Lagrangian representations are effective on other vector fields.
%
%A challenge for Lagrangian analysis is 
%%
%Time-varying vector fields, however, are complicated in nature and predetermining uncertainty of tim
%
%Predicting the uncertainty of time-varying vector fields is challenging and remains an open research question.
%%
%Thus, the application of Lagrangian representations to a broad range of real-world simulation vector fields remains an open questions 
%
%
%The quality of reconstruction depends on the vector field as well as configuration specifics such as sampling strategy and frequency of storage. 
%
%
%
%However, many of hese studies been conducted in theoretical in situ environments
%What data sets have been considered -> analytical data sets, climate and ocean modeling data, SPH data, Gerris flow solver. 
%
%
%
%Many of these strategies 
%The majority ths studies have considered theoretical in situ environments and operated 
%
%
%Existing evaluations of reduced Lagrangian representations, although often conducted in theoretical in situ environments (data is loaded from disk), have been encouraging. 
%%
%and show that performance characteristics vary from case to case. 
%%
%The quality of reconstruction depends on the vector field as well as configuration specifics such as sampling strategy and frequency of storage. 
%%
%
%
%\fix{Could you add some citations for different app areas?}
%Agranovsky et al.~\cite{agranovsky2014improved} proposed computing reduced Lagrangian representations using in situ processing to access the complete spatiotemporal resolution of the simulation.
%
%More recently, Pascal et al.~\cite{envirvis.20171099,siegfried2019tropical} used embedded routines to compute reduced Lagrangian data in order to explore coastal upwelling activity and visualize a derived scalar field representing trajectory density.
%
%Existing evaluations of reduced Lagrangian representations, although often conducted in theoretical in situ environments using analytical data sets where ground truths are known, have been encouraging and show that performance characteristics vary from case to case. 
%%
%%Existing evaluations, although on a limited set of applications and primarily conducted in theoretical in situ settings, are promising.
%%
%\fix{I think we need a better reason that it being interesting to us.  Presumably about enabling these app areas to benefit from Lagrangian.}

\begin{figure}[!t]
\centering
\includegraphics[width=0.9\linewidth]{Images/sample.pdf}
\vspace{-4mm}
\caption{Notional space-time visualization of Lagrangian representations for a time-varying 1D flow. The black trajectories are computed in situ and encode the behavior of the vector field between start time \textit{t$_{s}$} and end time \textit{t$_{e}$}. In a post hoc setting, a Lagrangian-based advection scheme \textit{L}, i.e., a technique to interpolate the extracted data, is used to calculate the trajectory of a new particle p$_{1}$ . The red trajectory is the trajectory reconstructed post hoc and the blue trajectory is the ground truth. The end location of the red trajectory deviates by a margin of error from the ground truth. The quality of reconstruction often depends on the nature of the time-varying vector field.}
\vspace{-5mm}
\label{fig:sample}
\end{figure}


%
%However, existing evaluations of reduced Lagrangian representations are on a limited, and often analytical, set of applications, most set in theoretical in situ settings. %~\cite{agranovsky2014improved}\cite{sane2018revisiting}.
%
%However, in addition to only preliminary evaluations of efficacy on a limited, and often analytical, set of data, these studies are performed in theoretical settings.
%
%Thus, the use of reduced Lagrangian representations for a broader range of applications and in practice, remains not well understood.
%
%In this paper, we investigate Lagrangian representations for time-varying vector fields produced by cosmology and seismology simulations in representative HPC settings.
%

In this paper, our unique contribution is an investigation of Lagrangian representations to encode self-gravitating gas dynamics of a cosmology simulation and seismic wave propagation of a seisomology simulation.
%
%We measure the effectiveness of the technique by considering in situ encumbrance and post hoc efficacy.
%
%For the self-gravitating gas dynamics vector field, our experiments show that Lagrangian representations can enable accurate analysis of particle evolution for reduced data storage. 
%structure formation in the simulation.
%
%Further, we show that reconstruction accuracy statistically improves as temporal sparsity increases, providing an opportunity for increased data reduction.
%
%For the seismic wave propagation vector field, we find that Lagrangian representations are well suited to capture transient wave patterns in the flow, resulting in accurate reconstructions for several data reduction options.
%
For both applications, our experiments show that Lagrangian representations offer high data reduction, in many cases requiring less than 1\% storage of the complete time-varying vector fields, for a small loss of accuracy. 
%
Further, our study shows Lagrangian representations are viable to compute in representative HPC environments, requiring under 10\% of total execution time for data analysis and visualization in the majority of configurations tested. 
% and should be considered more widely by applications that can benefit from time-varying vector field analysis.
%Overall, our experiments show that Lagrangian representations offer high data reductions of the time-varying vector fields considered for a small loss of accuracy, are viable to compute in situ, and should be considered more widely for time-varying vector field analysis.  
%\fix{the only mention of viability is in this last sentence.  Most of it is about accuracy.  Could it become more symmetric?  Should this paragraph introduce the notion of how to evaluate Lagrangian efficacy?}
%accurate reconstruction of the time evolutions of the simulation  
%
%
%choices for both spatial and temporal sampling can impact the reconstruction
%
%
%We study the efficacy of Lagrangian representations by conducting a statistical evaluation across a range of spatiotemporal configurations as well as a qualitative evaluation for varying data reduction factors. %for cosmology and seismology applications.
%
%Our experiments evaluate viability by measuring in situ encumbrance in representative settings, i.e., a tightly-coupled integration with the simulation code and execution on HPC resources.
%
%Our results show that Lagrangian representations offer strong data storage-accuracy propositions for time-varying analysis of the vector fields studied and require a low cost to extract. %using HPC resources.
%
%Our experiments evaluate the cost of a tightly-coupled in situ processing integration with each simulation code using Ascent and execution on HPC resources.
%\item An evaluation of in situ encumbrance in representative settings, i.e., tightly-coupled integration with a simulation and use of HPC resources at scale.
%\item A benchmarking study to measure in situ encumbrance using Cloverleaf3D.
%\end{tightItemize}

%First, we explore the use of Lagrangian representations to encode particle transport behavior of baryonic matter evolving under self-gravitating gas dynamics in the Nyx cosmology simulation~\cite{almgren2013nyx}.
%
%Next, we evaluate the potential of Lagrangian representations to encode the behavior of a fourth-order seismic wave propagation simulation SW4~\cite{petersson2015wave}. 
%
%In addition to investigating the use of Lagrangian representations for these vector fields, our study improves on prior efficacy and performance evaluations.
%
%We present the first statistical analysis of reconstruction accuracy for a range of spatiotemporal configurations as well as the first qualitative comparison to the traditional Eulerian approach. 
%
%Finally, we measure performance in a representative setting by considering integrated in situ infrastructure and execution using both GPUs and CPUs on a supercomputer.

%toring the time-varying vector field using a reduced Lagrangian representation is 
%
%In this paper, we focus on the axes of viability and efficacy. 
%
%Prior studies of reduced Lagrangian flow maps have shown improved accuracy-storage propositions compared to the traditional Eulerian paradigm under sparse temporal settings.
%
%However, only theoretical in situ settings, i.e., data sets are loaded from disk, on a single compute node or at small scale have been considered.
%
%With respect to type of time-varying vector field, either analytical data sets or data from climate and ocean modeling simulations has been considered.
%
%Further, for evaluation of reconstruction accuracy, comparisons to the Eulerian paradigm have been limited to quantitative analysis using a single average error.
%

%To determine viability, it is essential to evaluate the cost of in situ Lagrangian analysis in relation to the simulation execution time. 
%%
%In situ computation of a Lagrangian flow map, unlike the majority of in situ analysis tasks, requires computation and transfer of control between simulation and in situ analysis task every single cycle.
%%
%Our study integrates in situ infrastructure supporting Lagrangian analysis with simulation codes and evaluates execution on CPUs and GPUs on a modern supercomputer.
%%
%To contribute to research surrounding efficacy of the technique, we first consider three Exascale Computing Project simulation codes: a mini-application used for benchmarking, a wave propagation seismology simulation, and a Lyman-Alpha forest cosmology simulation, thus expanding our understanding across a variety of vector fields.
%%
%Next, we evaluate reconstruction accuracy by quantitatively measuring distribution of error across various configurations and present the first qualitative evaluation of the technique. 
%
%Overall, our study considers the current state-of-the-art of in situ Lagrangian analysis in representative high-performance computing settings, for two real-world simulation codes, with improved evaluations of viability and efficacy.
%%
%We believe this study is valuable to demonstrate the usage of the technique and serves to encourage adoption beyond theoretical research and climate and ocean modeling.
