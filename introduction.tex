%Topics for introduction / related work:
%1. One of the challenges faced by scientists is the data analysis and visualization of the large amounts of data generated by simulation codes on HPC resources.
%2. In situ processing offers a potential solution by circumventing the need to store all the raw data. 
%3. Most in situ analysis tasks operate infrequently or are triggered. Computing Lagrangian flow maps is unlike other analysis tasks and requires computation every cycles of the simulation - this places an overhead on the simulation code and raises questions of viability. 
%4. Lagrangian analysis has been extensively studied for ocean modeling in an offline setting and now efforts to support online analysis are being made.
%5. Reduced Lagrangian flow maps have been shown to be superior to traditional Eulerian subsampling for primarily analytical data sets, in theoretical settings or 2D flows, and compared using a single average error across all post hoc interpolated particles. 
%6. We believe reduced Lagrangian flow maps need to be explored for more real world time-varying vector fields to better understand their applicability, and more closely evalauted - both quantitatively and qualitatively - to understand efficacy characteristics.
%7. We strongly believe Lagrangian flow maps need to be evaluated in representative settings - integrated with a simulation code and executing on a supercomputer - to provide insight into viability.
%8. Given reduced Lagrangian flow maps are a new paradigm - it opens research opportunities along multiple axes - namely, sampling strategy, post hoc interpolation, ease of use (integration), performance (viability) and applicability (configuration settings, vector field type, etc). 
%9. Several works have advanced research along these axes - but none have considered viability in relation to simulation execution times or application to seismology or cosmology vector fields. Demonstration in this form we believe encourages wider adoption of the paradigm beyond the theoretical level.
%
%
%
%
High-performance computing resources play a key role in advancing computational science.
%
Although these machine enable performing scientific simulations at high spatiotemporal resolutions for accurate modeling, the total data generated can be prohibitively large.
%
Compromise in the form of storing a subset of the data can impact the extent and accuracy of subsequent post hoc exploratory analysis and visualization.
%
In particular, accurate time-varying vector field analysis and visualization requires access to the full spatiotemporal resolution.
%
Since storing the entire simulation output is very expensive, scientists resort to temporal subsampling, lossy compression, or limit analysis to individual time slices.
%
An emerging paradigm to address large data challenges is the use of \textit{in situ} processing to perform runtime analysis/visualization or data reduction.
%
%

%In situ Lagrangian analysis is an emerging paradigm to enable post hoc exploration of time-varying vector fields.
% and has presented research opportunities along multiple orthogonal axes.
%
Lagrangian analysis is a powerful tool to study time-varying vector fields and has been widely employed for ocean modeling applications~\cite{VANSEBILLE201849}.
%
The notion of calculating a Lagrangian representation or \textit{flow map}, i.e., sets of particle trajectories, ``online'' (in situ) for ``offline'' (post hoc) exploration was proposed by Vries et al.~\cite{vries2001calculating} for an ocean modeling simulation.
%
Figure~\ref{fig:sample} illustrates the approach.
%
The black trajectories are extracted online and are later used to reconstruct the red trajectory offline.
%
The end location of the red trajectory is computed using a Lagrangian interpolation scheme \textit{L} and deviates by a margin of error from the ground truth.
%
The quality of reconstruction depends on the vector field as well as configuration specifics such as sampling strategy and frequency of storage. 
%
Agranovsky et al.~\cite{agranovsky2014improved} proposed computing reduced Lagrangian representations using in situ processing to access the complete spatiotemporal resolution of the simulation.
%
More recently, Pascal et al.~\cite{envirvis.20171099,siegfried2019tropical} use embedded routines to compute reduced Lagrangian data in order to explore coastal upwelling activity and visualize a derived scalar field representing trajectory density.
%
The use of reduced Lagrangian representations across a wide range of applications and in practice, however, remains not well understood.
%
In this paper, we investigate in situ data reduction via Lagrangian representations for time-varying vector fields produced by cosmology and seismology simulations in representative high-performance computing settings.
%

\begin{figure}[!t]
\centering
\includegraphics[width=0.9\linewidth]{Images/sample.pdf}
\vspace{-5mm}
\caption{Notional space-time visualization of using Lagrangian representations in a 1D flow for exploring time-dependent vector fields. The black trajectories are computed in situ and encode the behavior of the vector field between start time \textit{t$_{s}$} and end time \textit{t$_{e}$}. In a post hoc setting, a Lagrangian-based advection scheme \textit{L} is used to interpolate the extracted data and calculate the trajectory of a new particle p$_{1}$ . The red trajectory is the reconstructed trajectory and the blue trajectory is the ground truth.}
\vspace{-5mm}
\label{fig:sample}
\end{figure}

First, we explore the use of Lagrangian representations to encode particle transport behavior of baryonic matter evolving under self-gravitating gas dynamics in the Nyx cosmology simulation~\cite{almgren2013nyx}.
%
Next, we evaluate the potential of Lagrangian representations to encode the behavior of a fourth order seismic wave propagation simulation SW4~\cite{petersson2015wave}. 
%
In addition to investigating the use of Lagrangian representations for these vector fields, our study improves on prior efficacy and performance evaluations.
%
We present the first statistical analysis of reconstruction accuracy for a range of spatiotemporal configurations as well as the first qualitative comparison to the traditional Eulerian approach. 
%
Finally, we measure performance in a representative setting by considering integrated in situ infrastructure and execution using both GPUs and CPUs on a supercomputer.

%toring the time-varying vector field using a reduced Lagrangian representation is 
%
%In this paper, we focus on the axes of viability and efficacy. 
%
%Prior studies of reduced Lagrangian flow maps have shown improved accuracy-storage propositions compared to the traditional Eulerian paradigm under sparse temporal settings.
%
%However, only theoretical in situ settings, i.e., data sets are loaded from disk, on a single compute node or at small scale have been considered.
%
%With respect to type of time-varying vector field, either analytical data sets or data from climate and ocean modeling simulations has been considered.
%
%Further, for evaluation of reconstruction accuracy, comparisons to the Eulerian paradigm have been limited to quantitative analysis using a single average error.
%

%To determine viability, it is essential to evaluate the cost of in situ Lagrangian analysis in relation to the simulation execution time. 
%%
%In situ computation of a Lagrangian flow map, unlike the majority of in situ analysis tasks, requires computation and transfer of control between simulation and in situ analysis task every single cycle.
%%
%Our study integrates in situ infrastructure supporting Lagrangian analysis with simulation codes and evaluates execution on CPUs and GPUs on a modern supercomputer.
%%
%To contribute to research surrounding efficacy of the technique, we first consider three Exascale Computing Project simulation codes: a mini-application used for benchmarking, a wave propagation seismology simulation, and a Lyman-Alpha forest cosmology simulation, thus expanding our understanding across a variety of vector fields.
%%
%Next, we evaluate reconstruction accuracy by quantitatively measuring distribution of error across various configurations and present the first qualitative evaluation of the technique. 
%
%Overall, our study considers the current state-of-the-art of in situ Lagrangian analysis in representative high-performance computing settings, for two real-world simulation codes, with improved evaluations of viability and efficacy.
%%
%We believe this study is valuable to demonstrate the usage of the technique and serves to encourage adoption beyond theoretical research and climate and ocean modeling.
