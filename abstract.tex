Although a large number of applications produce time-varying vector fields, many limit subsequent analysis to single time slices due to constraints on data storage.
%
In recent years, reduced Lagrangian representations of time-varying vector fields have been researched as a potential approach to enable accurate post hoc analysis.
%
The utility of this technique, however, varies based on the nature of the vector field, and the configuration specifics that determine computation costs as well as storage requirements.
%In recent years, reduced Lagrangian representations of time-varying vector fields have been considered to enable accurate post hoc analysis.
%
This paper contributes the investigation of Lagrangian representations computed in situ for two application domains: cosmology and seismology.
%
%These applications generate time-varying vector fields that vary in nature to those previously studied.
%
To investigate utility, we evaluate effectiveness and viability. 
%
To inform effectiveness, our study conducted a statistical analysis of efficacy across a range of spatiotemporal configurations as well as a qualitative evaluation. %comparison of the technique to the traditional Eulerian approach.
%
To inform viability, we considered representative HPC environments, integrated in situ infrastructure with the simulation codes, and performed Lagrangian in situ reduction using GPUs as well as CPUs.
%
We found that time-varying vector fields produced by the cosmology and seismology applications considered can be reduced by 200X via Lagrangian representations, while maintaining accurate reconstruction and requiring under 10\% of total execution time in 11 of 13 experiments.
%In particular, we found Lagrangian representations for data generated by the Nyx cosmology simulation are sensitive to both spatial and temporal sampling, notably providing higher accuracy as temporal sparsity increases.
%
%For the SW4 seismology simulation, we found Lagrangian representations can accurately encode time-varying vector fields for seismic wave propagation and offer strong data reduction propositions.
%
%Additionally, we include results from a benchmarking study using an ECP mini-application that utilizes heterogenous compute resources.
%
%Overall, the in situ processing cost to compute Lagrangian representations was under 10\% of total execution time in 11 of 13 experiments.
%

\fix{HANK:}
Although many types of computational simulations produce time-varying vector fields, 
subsequent analysis is often limited to single time slices due to excessive costs.
%
Fortunately, a new approach using a Lagrangian representation can 
provide time-varying vector field analysis while mitigating these costs.
%
With this approach, a Lagrangian representation is calculated while the simulation code is running, and the result is explored after the simulation.
%
Importantly, 
 the efficacy of this approach varies based on the nature of the vector field, 
requiring fresh investigation for each application area.
%
With this study, we consider efficacy for cosmology and seismology
applications, which have never been previously been demonstrated.
%
We do this by considering encumbrance (on the simulation to calculate the
Lagrangian representation) and also accuracy (of the reconstructed result).
%
To inform encumbrance, we 
integrated in situ infrastructure with two simulation codes, 
and evaluated on representative HPC environments,
performing Lagrangian in situ reduction using GPUs as well as CPUs.
%
To inform accuracy, our study conducted a statistical analysis across a range of spatiotemporal configurations as well as a qualitative evaluation.
In all, we demonstrate efficacy for both cosmology and seismology --- time-varying vector fields from these domains can be reduced by 200X via Lagrangian representations, while maintaining accurate reconstruction and requiring under 10\% of total execution time in over 80\% of our experiments.
