A large number of applications could benefit from time-varying vector field analysis, however, most limit analysis to single time slices due to constraints on data storage.
%
In recent years, reduced Lagrangian representations of time-varying vector fields have been considered to enable accurate post hoc analysis.
%
%At present, the use of reduced Lagrangian representations has been explored for only a limited set of applications.
%
%Further, prior evaluations of efficacy have been preliminary and demonstrated in theoretical settings. 
%
This paper investigates the use of Lagrangian representations computed in situ for two application domains: cosmology and seismology.
%
%These applications generate time-varying vector fields that vary in nature to those previously studied.
%
Our study contributes the first statistical analysis of efficacy across a range of spatiotemporal configurations as well as the first qualitative comparison of the technique to the traditional Eulerian approach.
%
By considering representative high-performance computing environments and integrating in situ infrastructure with the simulation codes, our study demonstrates that in situ reduction via Lagrangian representations can be performed viably using GPUs as well as CPUs.
%
In particular, we found that use of Lagrangian representations for data generated by the Nyx cosmology simulation is sensitive to temporal sampling, providing higher accuracy as temporal sparsity increases.
%
For the SW4 seismology simulation, we find that Lagrangian representations can accurately encode time-varying vector fields for seismic wave propagation and offer strong data reduction propositions.
%
Overall, we were able to compute Lagrangian representations while requiring under 10\% of execution time for in situ processing in 11 of 13 configurations considered.
%
