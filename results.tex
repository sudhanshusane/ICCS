We organize our discussion of results by application. 
%
Tables~\ref{table:nyx_encumbrance} and~\ref{table:sw4_encumbrance} provide information pertaining to in situ encumbrance such as concurrency information, spatial dimensions, Sim$_{cycle}$, number of particles, \textbf{InSituMem}, \textbf{Step}, and \textbf{DAV\%}, for each application. 
%
Figures~\ref{fig:nyx_violinplot},~\ref{fig:nyx_figure},~\ref{fig:sw4_violinplot}, and ~\ref{fig:sw4_figure} show the results of our post hoc efficacy evaluation.
%
For each application, the figures are annotated with configuration specifics such as the \textbf{DS}, \textbf{1:X}, and \textbf{I}.
%
Further, results from Lagrangian and Eulerian tests are distinguished explicitely in the captions or are labeled L$T$ and E$T$ respectively, where $T$ is the test number.
%
We refer to these tables and figures in our discussion that follows. 

\subsection{Nyx Cosmology Simulation}
\subsubsection{In Situ Encumbrance}
Using all the cores of two CPUs on a single compute node, we use OpenMP to parallelize the Nyx simulation and Lagrangian VTK-m filter.
%
We test 2 options for grid size - $69^{3}$ and $129^{3}$ - on a single rank, and 3 particle advection workloads (1:1, 1:8, 1:27) each.
%
In a single compute node hour, the simulation performs 300 and 38 cycles approximately, when using $69^{3}$ and $129^{3}$ grid sizes respectively.
%
%In a single compute node hour, the simulation performs approximately 300 cycles for the $69^{3}$ resolution and approximately 38 cycles for the $129^{3}$ resolution.
%
An 8X increase in grid size results in a proportional increase in Sim$_{cycle}$ but only a small increase in particle advection costs for the same number of particles.
%
We expect a single rank to operate on between $32^{3}$ to $256^{3}$ grid resolutions, thus our workloads provide a representative estimate of \textbf{DAV\%}.
%
%For example, sampling a $256^{3}$ using a 1:8 reduction involves computing 2.1M basis trajectories. 

An encouraging finding is the low in situ encumbrance when performing L-ISR on the CPUs.
%
As an increasingly nuymber of computationals simulations target execution on GPUs, future work can consider offloading L-ISR computation to CPUs.
%
Overall, considering the longer Sim$_{cycle}$ times for the Nyx simulation, and parallel compution coupled with low memory latency when using CPUs, the highest in situ encumbrance to extract a Lagrangian represenation was 0.1\% of the simulation time or under 0.06s to compute 2.1M basis trajectories per cycle.
\input{timings_nyx.tex}
\begin{figure}[!t]
\centering
\includegraphics[width=\linewidth]{Images/nyx_violinplot.pdf}
\vspace{-5mm}
\caption{Accuracy results for reconstructing the Nyx time-varying vector field using 4 Eulerian configurations and 12 Lagrangian configurations. The violin plots encode the distribution of the particle reconstruction error across tests. While Eulerian configurations contain greater uncertainty as the value of \textbf{I}, i.e., temporal sparsity, increases, the Lagrangian representations increase in accuracy. For Lagrangian representations, we find high reconstruction accuracy relies on a high spatial sampling resolution as well.}
\vspace{-5mm}
\label{fig:nyx_violinplot}
\end{figure}

\begin{figure}[!t]
\centering
\includegraphics[width=\linewidth]{Images/nyx_figure.pdf}
\vspace{-5mm}
\caption{
\fix{THERE IS A FORMATTING ERROR WITH VSPACE BELOW}
Pathline visualization of baryonic particles evolution in self-gravitating gas dynamics of Nyx simulation. Using 10,000 randomly seeded particles, we visualize pathlines over 300 cycles. To focus on regions where particles cluster to form dense regions, we set opacity of the pathline to be directly proportional to time. Thus, we are able to focus on clustering as well as provide context of transport toward these regions. Lagrangian representations are able to reconstruct the ground truth trajectories and capture clustering accurately when high spatial sampling is used~(1:1, 1:8). However, when using a 1:27 data reduction factor, some clusters are visualized less clearly.} 
\vspace{-5mm}
\label{fig:nyx_figure}
\end{figure}


\subsubsection{Post Hoc Efficacy}
To evaluate the usefulness of Lagrangian representations to encode time-varying self-gravitating gas dynamics, we considered 3 options for data reduction (1:1, 1:8, 1:27) and 4 options for \textbf{I}~(25, 50, 100, 200).
%
We construct pathlines for 50,000 randomly placed particles over 400 cycles during post hoc analysis.
%
%We measure the error of reconstruction of each pathline by comparing to ground truth trajectories.
%
%Additionally, we compute pathlines using Eulerian representations at full spatial resolution and same options for \textbf{I}.
%
We visualize the distribution of reconstruction error for all tests in Figure~\ref{fig:nyx_violinplot}.
%We measure the error or reconstruction of each pathline. and visualize the distribution in Figure~\ref{} along with  
%

The self-gravitating gas dynamics of this simulation produce a vector field that captures the transport of randomly distributed particles to multiple high density clusters.
%
Particles travel with increasing velocity as clusters increase in density.
%
For this case, we find that Eulerian temporal subsampling performs better for small values of \textbf{I}.
%
This can be expected as reconstruction using an Eulerian representation and fourth-order Runge Kutta interpolation remain more accurate than second-order barycentric coordinates interpolation employed to interpolate Lagrangian representations~\cite{bujack2015lagrangian}\cite{hummel2016error}.
%
%This can be expected for two reasons: 1) for small values of \textbf{I}, trajectories computed using fourth-order RK4 and Eulerian representations can remain more accurate than second-order barycentric coordinates interpolation and Lagrangian representations (although these are computed in situ using fourth-order RK4), and 2) use of several short Lagrangian flow maps can result in a higher error propagation and accumulation~\cite{bujack2015lagrangian,hummel2016error}.
%
However, as the value of \textbf{I} increases, the distribution of error for the Lagrangian tests indicates that a larger percentage of samples are reconstructed more accurately.
%
In contrast, Eulerian representations become less effective at reconstructing the vector field due to increased numerical approximation and accumulating error.
%
The Lagrangian representation computed in situ using RK4 encodes the behavior over an interval of time more accurately for the majority of samples.
%
%With regard to spatial sampling resolution, for the grid size we consider, we find that reducing the number of samples has a considerable impact on the efficacy of reconstruction.
%
%This indicates that small changes in starting location can result in particles transporting to different clusters.
%

We use pathlines with manually set transfer functions to visualize the evolution and clustering of particles in regions of high density.
%
The total size of the simulation vector field data used to compute the ground truth is 5.3GB.
%
We visualize a random subset of 10,000 pathlines in Figure~\ref{fig:nyx_figure} for configurations with \textbf{I} set to 25.
%To visualize the evolution and clustering of particles in regions of high density, we visualize a random subset of 10,000 pathlines in Figure~\ref{fig:nyx_figure} for configurations with \textbf{I} set to 25.
%
We note that Lagrangian configurations (L04 to L12) using larger values of \textbf{I} are more accurate statistically~(Figure~\ref{fig:nyx_violinplot}).
%
%
The Lagrangian representations demonstrate the ability to closely reconstruct regions where dense clusters are formed while requiring a fraction of the total simulation data size.
%
For example, the 1:8 Lagrangian configuration enables the visualization of transport to dense clusters while requiring only 27MB, i.e., a 200X data reduction of the uncompressed vector field.
%

\subsection{SW4 Seismology Simulation}
\subsubsection{In Situ Encumbrance}
\input{timings_sw4.tex}
For the SW4 simulation, we consider five grid sizes at varying concurrencies.
%
In each case, we used all 6 GPUs available on a compute node to execute the simulation and L-ISR.
%
For all L-ISR workloads tested, the execution time required per cycle remained under 1 second on average and the maximum in situ memory required by a node was 112 MB to compute the trajectories of 4.4M particles.
%
DAV\% was closely related to both the Sim$_{cycle}$ and \textbf{Step} cost.
%
Using fixed concurrency, fixed data reduction factor, and varying grid sizes, we observed DAV\% decrease from 11.67\% to 4.365\% as the grid size increased.
%
Using fixed concurrency, fixed grid size, and varying data reduction factor, we observed DAV\% increase from 1.201\% to 6.175\% as the number of particles increased.
%

\begin{figure}[!t]
\begin{subfigure}{0.495\textwidth}
\centering
\includegraphics[width=\linewidth]{Images/sw4_violinplot1.pdf}
\vspace{-5mm}
\caption{High displacement near the epicenter.}
\label{fig:epicenter}
\end{subfigure}
\begin{subfigure}{0.495\textwidth}
\centering
\includegraphics[width=\linewidth]{Images/sw4_violinplot2.pdf}
\vspace{-5mm}
\caption{Low displacement away from the epicenter.}
\label{fig:clusters}
\end{subfigure}
\vspace{-2mm}
\caption{Violin plots of distribution of particle displacement for the ground truth (GT), 1 Eulerian configuration and 4 Lagrangian configurations. The Eulerian configuration with access to limited number of time slices, overestimates the displacement. The Lagrangian representation encoding time intervals more accurately capture displacement in both setttings, in regions near and away from the epicenter.}
\vspace{-5mm}
\label{fig:sw4_violinplot}
\end{figure}

\begin{figure}[!t]
\centering
\includegraphics[width=\linewidth, trim={1cm, 0cm, 0.9cm, 0cm}, clip]{Images/sw4_figure_small.pdf}
\vspace{-5mm}
\caption{Visualization of the displacement field derived from reduced Lagrangian representations near the epicenter using multiple isosurfaces. The ground truth is computed using 2000 cycles of the seismic wave propagation simulation. Although at higher data reduction factors regions of high displacement are underestimated, Lagrangian representations are capable of accurately reconstructing the overall feature structure.} 
\vspace{-5mm}
\label{fig:sw4_figure}
\end{figure}


\subsubsection{Post Hoc Efficacy}
We study the reconstruction of the time-varying displacement vector field encoding wave propagation by considering 4 options for data reduction~(1:1, 1:8, 1:27, 1:64) and 1 option for \textbf{I}~(250).
%
The ground truth is computed using 2000 simulation cycles and requires 245 GB.
%
The displacement is highest near the epicenter and reduces as waves propagate further away.
%
For each simulation run, we measure the displacement of 200,000 samples reconstructed near the epicenter (Figure~\ref{fig:epicenter}) and 90,000 samples reconstructed in six regions away from the epicenter (Figure~\ref{fig:clusters}).
%
We provide the ground truth (GT) distribution of displacement.
%
In both cases, Lagrangian representations offer significant data reduction while maintaining high accuracy.
%
We find that as the number of basis trajectories extracted reduces, the displacement for some samples can be underestimated. 
%
In contrast, using a temporally subsampled Eulerian representation (E01) results in significant over-estimation of displacement.
%
This can be expected as temporal subsampling fails to capture the transient nature of wave propagation, while Lagrangian representations encoding behavior over an interval of time remain accurate.
%
Compared to Figure~\ref{fig:epicenter}, samples in Figure~\ref{fig:clusters} have smaller displacement and multimodal distribution that is the result of samples collected from six regions of the domain away from the epicenter. 
%

Figure~\ref{fig:sw4_figure} visualizes the displacement field near the epicenter using multiple semi-opaque isosurfaces.
%
Although most regions are well reconstructed using Lagrangian representations, we find that regions of highest displacement are underestimated as the data reduction factor increases.
%
Overall, we find that Lagrangian representations offer high data reduction options for small loss of accuracy and should be considered more widely for seismic wave propagation vector fields.

