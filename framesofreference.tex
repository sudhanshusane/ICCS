\vspace{-1mm}
\subsection{Frames of Reference}
%
In fluid dynamics, there are two frames of reference to observe fluid motion: Eulerian and Lagrangian.
%
With the Eulerian frame of reference, the observer is in a fixed position.
%
With the Lagrangian frame of reference, the observer is attached to a fluid parcel and is moving through space and time.

%
Storage of a flow field in an Eulerian representation is typically done by means of its velocity field.
%
A velocity field $v$ is a time-dependent vector field that maps each point $x\in \mathbb R^d$ in space to the velocity of the flow field for a given time $t\in \mathbb R$
%
\begin{eqnarray}
{v} : \mathbb R^d \times \mathbb R \to \mathbb R^d,\; x,t \mapsto v(x,t)
\end{eqnarray}

%
In a practical setting, a flow field at a specific time/cycle is defined as vector data on a fixed, discrete mesh.
%
Time-varying flow is represented as a collection of such data over a variety times/cycles.


Storage of a flow field in a Lagrangian representation is done by means of its flow map $F_{t_0}^{t}$.
%
The flow map is comprised of the starting positions of massless particles $x_0$ at time $t_0$ and their respective trajectories that are interpolated using the time-dependent vector field.
%
Mathematically, a flow map is defined as the mapping
\begin{eqnarray}
F_{t_0}^{t}(x_0):\mathbb R \times \mathbb R \times \mathbb R^d \to \mathbb R^d,\; t \times t_0 \times x_0 \mapsto F_{t_0}^{t}(x_0) = x(t)
\end{eqnarray}
%
of initial values $x_0$ to the solutions of the ordinary differential equation
%
\begin{eqnarray}
\frac{d}{dt}x(t) = v(x(t),t)
\end{eqnarray}

In a practical setting, the flow map is represented as sets of particle trajectories calculated in the time interval $[t_0,t]\subset \mathbb R$.
%
The stored information, encoded in the form of known particle trajectories (i.e., a Lagrangian representation), encodes the behavior of the time-dependent vector field over an interval of time.
%
